%% ELEMENTOS PRE-TEXTUAIS

%% Capa
\inserecapa

%% Folha de rosto
\inserefolhaderosto


%% Ficha catalografica. AO IMPRIMIR, DEIXAR NO VERSO DA FOLHA DE ROSTO.
\inserecatalog  


%% Folha de aprovacao
\begin{folhadeaprovacao}

  \begin{center}
    {\chapterfont \MakeUppercase{\bfseries \insereautor}}

    \vfill
    \begin{center}
      {\chapterfont \MakeUppercase{\bfseries\inseretitulo \inseresubtitulo}}
    \end{center}
    \vfill
    
    \hspace{.45\textwidth}
    \begin{minipage}{.5\textwidth}
        \inserenatureza
        \\ \\
        \begin{center}COMISSÃO EXAMINADORA \end{center}
         \assinatura{Prof. Msc. \insereorientador \ - Orientador \\ Centro Universitário -- Católica de Santa Catarina} 
      %  \assinatura{Professor Dr. \inserecoorientador \ - Coorientador \\ Universidade Federal de Juiz de Fora}
         \assinatura{Professor Dr. ?? \\ Universidade ???}
         \assinatura{Professor Dr. ?? \\ Universidade ??} 
    \end{minipage}%
    \vfill
   \end{center}
           

%  \assinatura{...} %%RETIRE O % E PREENCHA SE PRECISAR
%  \assinatura{...}
%  \assinatura{...}
\end{folhadeaprovacao}


%% Dedicatoria. OPCIONAL. Retirar o ``%'' de cada das 4 linhas abaixo, caso queira.
% \begin{dedicatoria} \vspace*{\fill} \centering \noindent
%   \textit{ Dedico este trabalho ... (opcional)} 
%   \vspace*{\fill}
% \end{dedicatoria}


%% Agradecimentos. OPCIONAL. CASO SEJA BOLSISTA, INSERIR OS DEVIDOS AGRADECIMENTOS.
%\begin{agradecimentos}

%Este trabalho é decicado à minha família, meus amigos, meu orientador e a Católica de Santa Catarina. 

%\end{agradecimentos}

%% Epigrafe. OPCIONAL
% \begin{epigrafe}
%     \vspace*{\fill}
% 	\begin{flushright}
% 		``Texto em que o autor apresenta uma cita\c{c}\~ao, seguida de autoria, relacionada com a                       
%   mat\'eria tratada no corpo do trabalho'' \\
% (ASSOCIA\c{C}\~AO BRASILEIRA DE NORMAS T\'ECNICAS, 2011, p. 2) \\
%   A ep\'igrafe elaborada conforme NBR 10520 (Ep\'igrafe - Opcional)
% 	\end{flushright}
% \end{epigrafe}


%% RESUMOS

%% Resumo em Portugu^es. OBRIGATORIO.
\setlength{\absparsep}{18pt} 
\begin{resumo}
 ... Resumo
\textbf{Palavras-chave}: Alguma Palavra-chave, Engenharia de Software. %finalizadas por ponto e inicializadas por letra maiuscula.

\end{resumo}
 
 
%% Resumo em Ingle^s
\begin{resumo}[ABSTRACT]
 \begin{otherlanguage*}{english}
 ... Abstract
\textbf{Keywords}: Some Keyword, Software Engineering.
 \end{otherlanguage*}
\end{resumo}

%% Seguindo o mesmo modelo acima, pode-se inserir resumos em outras linguas. 

%% Lista de ilustracoes. OPCIONAL.
\pdfbookmark[0]{\listfigurename}{lof}
\listoffigures*
\cleardoublepage


%% Lista de tabelas. OPCIONAL. Retire o ``%'' de cada das 3 linhas seguintes, caso queira.
\pdfbookmark[0]{\listtablename}{lot}
\listoftables*
\cleardoublepage

%% Lista de abreviaturas e siglas. OPCIONAL
\begin{siglas} %%ALTERAR OS EXEMPLOS ABAIXO, CONFORME A NECESSIDADE
  \item[SIGLA] Siglas vão aqui
\end{siglas}

%% Lista de simbolos. OPCIONAL
% \begin{simbolos} %%ALTERAR OS EXEMPLOS ABAIXO, CONFORME A NECESSIDADE
%   \item[$ \forall $] Para todo
%   \item[$ \in $] Pertence
%  \end{simbolos}

 
%% Sumario
\pdfbookmark[0]{\contentsname}{toc}
\tableofcontents*
\cleardoublepage

%% ----------------------------------------------------------

%% ELEMENTOS TEXTUAIS