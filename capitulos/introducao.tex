\setcounter{page}{1}
\chapter{INTRODU\c{C}\~AO}  %%Nesta linha, dentro de { }, digita-se em CAIXA ALTA, como apresentado aqui
\label{chap:01}

O \textit{software} está inserido em praticamente todos os aspectos de nossas vidas, podendo colaborar para a otimização da linha de produção de industrias, melhorar a gestão de empresas ou até mesmo mudar a forma como as pessoas se relacionam. Diante desta grande demanda, o \textit{software} e todos os artifícios envolvidos em seu desenvolvimento precisam ser constantemente aperfeiçoados, desta forma, a engennharia de \textit{software} surgiu com o objetivo de aplicar princípios de engenharia na especificação, no desenvolimento e na manutenção de \textit{software}.

A grande demanda para construções de \textit{softwares} cada vez mais complexos e com mais funcionalidade, acarretou em projetos extensos, com bases de códigos enormes, que desencadeiam diverssos problemas para o gerenciamento, por exemplo, erros de estimativa, custos elevados e até mesmo desgastes emocionais e desmotivação dos engenheiros envolvidos. Para minimizar os problemas de projetos extensos, a engennharia de \textit{software} possui o conceito de modularidade, nele é enfatizado que o \textit{software} modular pode ser melhor gerenciado que o monolítico. Além disso, a modularização possibilita que o \textit{software} seja planejado mais facilmente e que os custos para mautenção sejam reduzidos.

Diferente da abordagem monolítica o desenvolvimento de \textit{software} baseado em \textit{microservices} visa construir vários \textsit{softwares} menores, com propósitos específicos, que se comunicam entre sí para resolver problemas complexos, básicamente neste modelo a modularização é intencificada visando simplificar ao máximo a complexidade dos projetos. 