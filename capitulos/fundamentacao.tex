\chapter{FUNDAMENTAÇÃO TEÓRICA}
\label{chap:02}

Neste capítulo são abordadas as tecnologias e conceitos necessários para a compreensão e embasamento do objetivo proposto.

\section{MICROSERVICES}

\section{AUTOMAÇÃO DE TESTES}

Para garantir a qualidade dos \textit{microservices}, os mesmos devem ser amplamente testados. Normalmente no processo de desenvolvimento de \textit{software} existe uma fase de testes, que após a codificação, visa a garantia de qualidade e que o código gerado esteja funcionando como o esperado. Entretanto, com o crescimento do \textit{software} e com o aumento de sua complexidade, tornam a etapa de testes custosas. Dessa forma, a automação de testes surge como uma alternativa para minimizar os custos dessa etapa e agilizar o processo de desenvolvimento sem lesar a qualidade. Segundo \citeonline{bernardo2008importancia}, a grande vantagem desta abordagem, é que todos os casos de teste podem ser facilmente e rapidamente repetidos a qualquer momento e com pouco esforço.

De modo geral os testes automatizados são \textit{scripts} simples que exercitam funcionalidades do sistema e fazem verificações automáticas nos efeitos colaterais obtidos. A reprodutibilidade dos testes permite simular identicamente e inúmeras vezes situações específicas, garantindo que passos importantes não serão ignorados por falha humana e facilitando a identificação de um possível comportamento não desejado. \cite{bernardo2008importancia}

Com a finalidade de garantir a qualidade dos \textit{microservices} e garantir o contrato para comunicação entre os serviços, foram utilizados 2 tipos de testes automatizados: Testes unitários e testes de integração.

\subsection{Testes Unitários}

Para \citeonline{pressman}, um teste unitário focaliza o esforço de verificação na menor unidade de projeto de \textit{software}. Deste modo, por testar pequenas partes isoladas do \textit{software} os testes unitários podem ser de grande valor para a garantia de qualidade do código. Além disso, segundo \citeonline{bernardo2008importancia}, o teste de uma unidade é o tipo mais importante de teste para a grande maioria das situações, já que é ele que deve testar se um algoritmo faz o que deveria ser feito e garantir que o código encapsulado por uma unidade deve produzir o efeito colateral esperado.

\subsection{Testes de Integração}

\section{INTEGRAÇÃO CONTINUA}

\section{AUTOMAÇÃO DE DEPLOY}